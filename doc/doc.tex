\documentclass{article}
\usepackage[style = numeric, backend = biber]{biblatex}
\usepackage{graphicx, paralist, listings, fancybox, geometry, caption, floatrow, xcolor, color, colortbl, fancyhdr, amsmath, mathtools}
\usepackage{fouriernc}
\usepackage[T1]{fontenc}
\geometry{a4paper, top = 15mm, left = 5mm, right = 5mm, bottom = 17mm}
\fancypagestyle{plain}{
\fancyhead[L]{\texttt{GAME} documentation}
\fancyhead[R]{\textsc{\texttt{GAME} development team}}
\fancyfoot[C]{\thepage}
\addtolength\footskip{12pt}}
\definecolor{table_green}{rgb}{0, 0.6, 0}
\title{\texttt{GAME} Documentation}
\author{\texttt{GAME} Development Team}
\date{}
\newcommand{\md}[1]{\frac{D#1}{Dt}}
\newcommand{\omegabi}{\text{{\osgbi ω}}}
\newcommand{\mubi}{\text{{\osgbi μ}}}
\newcommand{\sigmabi}{\text{{\osgbi σ}}}
\newcommand{\epsilonbi}{\text{{\osgbi ϵ}}}
\newcommand{\etabi}{\text{{\osgbi η}}}
\newcommand{\zetabi}{\text{{\osgbi ζ}}}
\addbibresource{/home/max/my_texts/references.bib}
\DeclareFieldFormat[article]{title}{{#1}}

\begin{document}

\maketitle

The \textit{speed} $s$ of a model is defined by
%
\begin{eqnarray}
s \coloneqq \frac{\Delta t}{\Delta\tilde{t}},
\end{eqnarray}
%
where $\Delta t$ is the time step of the model and $\Delta\tilde{t}$ is the duration it takes to integrate from one time step to the next.

\section{NWP mode}
\label{sec:nwp_mode}

In NWP mode, one wants to be able to integrate one day within 8.5 minutes, which corresponds to $s = \frac{24\cdot 60}{8.5} = 169$. As a general rule, one can say that the computation time is in equal parts needed for data assimilation, the dynamical core and the processes involving moisture and radiation. Thus, the dynamical core has a mimimum speed of
%
\begin{eqnarray}
s \geq 509.
\end{eqnarray}

\subsection{Parallelization}
\label{sec:parallelization}

\section{Radiation scheme}
\label{sec:radiation_scheme}

GAME employs the so-called \texttt{RTE+RRTMGP (Radiative Transfer for Energetics + Rapid and Accurate Radiative Transfer Model for General Circulation Model Applications—Parallel)} \cite{doi:10.1029/2019MS001621}, \cite{rte-rrtmgp-github} scheme. It is bound to the C code through the API \texttt{RTE-RRTMGP-C} \cite{rte-rrtmgp-c-github}.

\appendix

\printbibliography

\end{document}













