% This source file is part of the Global Atmospheric Modeling Framework (GAME), which is released under the MIT license.
% Github repository: https://github.com/MHBalsmeier/game

\documentclass{article}
\usepackage{graphicx, paralist, listings, geometry, caption, floatrow, xcolor, color, colortbl, fancyhdr, amsmath, mathtools}
\usepackage{fouriernc}
\usepackage[T1]{fontenc}
\geometry{a4paper, top = 15mm, left = 5mm, right = 5mm, bottom = 17mm}
\fancypagestyle{plain}{
\fancyhead[L]{GAME handbook}
\fancyhead[R]{\textsc{GAME development team}}
\fancyfoot[C]{\thepage}
\addtolength\footskip{12pt}}
\definecolor{table_green}{rgb}{0, 0.6, 0}
\title{GAME Handbook}
\author{GAME Development Team}
\date{}
\newcommand{\md}[1]{\frac{D#1}{Dt}}
\newcommand{\omegabi}{\text{{\osgbi ω}}}
\newcommand{\mubi}{\text{{\osgbi μ}}}
\newcommand{\sigmabi}{\text{{\osgbi σ}}}
\newcommand{\epsilonbi}{\text{{\osgbi ϵ}}}
\newcommand{\etabi}{\text{{\osgbi η}}}
\newcommand{\zetabi}{\text{{\osgbi ζ}}}

\begin{document}

\maketitle

\section{Running the model}
\label{sec:running_the_model}

All physical quantities in this document are to be multiplied with their respective SI units.

\lstinputlisting[language = Bash, caption = {Example input file.}, frame = on]{jw_perturbed_moist.sh}\label{lst:input_file_example}

Listing \ref{lst:input_file_example} is an example of an input file. Table \ref{tab:input_file_explanation} explains the meanings of the variables.

\renewcommand{\arraystretch}{1.2}
\begin{table}
\centering
\begin{tabular}{|>{\centering}p{4.0 cm}|>{\centering}p{3 cm}|>{\centering}p{7 cm}|}
\hline \textbf{name} & \textbf{domain} & \textbf{meaning} \tabularnewline
\hline\hline operator & string & Operator of the model, for example \texttt{Company XYZ, Inc.} \tabularnewline
\hline overwrite\_run\_id & 0, 1 & if 0: use auto-generated run\_id, if 1: use manually set run\_id (see next line) \tabularnewline
\hline run\_id & string (optional) & run\_id to be used if overwrite\_run\_id is set to 1 \tabularnewline
\hline run\_span & integer & How long the model shall run into the future. \tabularnewline
\hline write\_out\_interval & integer $\geq$ 900 & Every how many seconds autput shall be generated. \tabularnewline
\hline grid\_props\_file & string & File name of the grid properties file. \tabularnewline
\hline init\_state\_filename & string & File name of the initialization state file. \tabularnewline
\hline init\_state\_file & string & Full path of the initialization state file. \tabularnewline
\hline output\_dir\_base & string & The directory to which output shall be written. \tabularnewline
\hline cfl\_margin & double & Manual reduction of the time step below the CFL criterion: $\Delta t = \left(1 - \text{cfl\_margin}\right)\Delta t^{(\text{CFL})}$. \tabularnewline
\hline diffusion\_on & 0, 1 & diffusion switch \tabularnewline
\hline dissipation\_on & 10, 1 & dissipation switch \tabularnewline
\hline tracers\_on & 0, 1 & tracers switch \tabularnewline
\hline rad\_on & 0, 1 & radiation switch \tabularnewline
\hline radiation\_delta\_t & double $\geq \Delta t$ & Every how many seconds the radiation flux densities shall be updated.  \tabularnewline
\hline write\_out\_mass\_dry\_integral & 0, 1 & Switch to decide wether a global integral of dry mass shall be written out at every time step. \tabularnewline
\hline write\_out\_entropy\_gas\_integral & 0, 1 & Switch to decide wether a global integral of the entropy shall be written out at every time step. \tabularnewline
\hline write\_out\_energy\_integral & 0, 1 & Switch to decide wether a global integral of the energy shall be written out at every time step. \tabularnewline
\hline 
\end{tabular}
\caption{Input file explanation.}
\label{tab:input_file_explanation}
\end{table}
\renewcommand{\arraystretch}{1}

\section{Grid generation procedure}
\label{sec:grid_generation_procedure}

A grid is determined by the following four properties:

\begin{itemize}
\item the resolution, specified via the parameter \texttt{RES\_ID}
\item the orography, specified via the parameter \texttt{ORO\_ID}
\item the height of the top of the atmosphere, specified via the parameter \texttt{TOA}
\item the number of layers, specified via the parameter \texttt{NUMBER\_OF\_LAYERS}
\end{itemize}

The grid generator needs to be recompiled for every specific resolution, therefore change the constant \texttt{RES\_ID} in the source file \texttt{grid\_generator.c} and execute the bash script \texttt{compile.sh}

\section{Test state generation procedure}
\label{sec:test_state_generation_procedure}

\end{document}













